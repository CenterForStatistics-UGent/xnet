\documentclass[jss]{jss}
\usepackage[utf8]{inputenc}

\providecommand{\tightlist}{%
  \setlength{\itemsep}{0pt}\setlength{\parskip}{0pt}}

\author{
FirstName LastName\\University/Company \And Joris Meys\\Ghent University
}
\title{Analysing Bipartite Networks With Two-Step Kernel Ridge Regression: The
R Package \pkg{xnet}}

\Plainauthor{FirstName LastName, Joris Meys}
\Plaintitle{Analysing Bipartite Networks With Two-Step Kernel Ridge Regression: The
R Package xnet}
\Shorttitle{\pkg{xnet}: Crossvalidating Two-Step Kernel Ridge Regression}

\Abstract{
This paper presents the R package \pkg{xnet} for cross-network analysis
of bipartite networks, using two-step kernel ridge regression. It uses
the crossvalidation shortcuts proposed by \citet{stock2018} to allow for
computationally efficient evaluation of the models based on a variety of
leave-one-out methods. The package provides functions for easy tuning,
fitting and evaluation of two-step kernel ridge regression in the
context of cross-network analysis. We illustrate the use of the
\pkg{xnet} package with datasets from different areas of research.
}

\Keywords{keywords, not capitalized, \proglang{R}}
\Plainkeywords{keywords, not capitalized, R}

%% publication information
%% \Volume{50}
%% \Issue{9}
%% \Month{June}
%% \Year{2012}
%% \Submitdate{}
%% \Acceptdate{2012-06-04}

\Address{
    FirstName LastName\\
  University/Company\\
  First line Second line\\
  E-mail: \email{name@company.com}\\
  URL: \url{http://rstudio.com}\\~\\
    }

% Pandoc header

\usepackage{amsmath}

\begin{document}

\hypertarget{introduction}{%
\section{Introduction}\label{introduction}}

This template demonstrates some of the basic latex you'll need to know
to create a JSS article.

\hypertarget{code-formatting}{%
\subsection{Code formatting}\label{code-formatting}}

Don't use markdown, instead use the more precise latex commands:

\begin{itemize}
\item
  \proglang{Java}
\item
  \pkg{plyr}
\item
  \code{print("abc")}
\end{itemize}

\hypertarget{r-code}{%
\section{R code}\label{r-code}}

Can be inserted in regular R markdown blocks.

\begin{CodeChunk}

\begin{CodeInput}
R> x <- 1:10
R> x
\end{CodeInput}

\begin{CodeOutput}
 [1]  1  2  3  4  5  6  7  8  9 10
\end{CodeOutput}
\end{CodeChunk}

\bibliography{refs.bib}


\end{document}

